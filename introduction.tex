\chapter*{Introduction}
\addcontentsline{toc}{chapter}{Introduction}

The traveling salesman problem (which will be denoted by \TSP{} for brevity's sake) is a classic problem in computer science. It is a typical example of a combinatorial optimization problem, that is, an optimization problem with a \emph{discrete} solution space.

In its simplest form, the \TSP{} asks the following question: ``A salesman wants to take \emph{the best\footnote{Usually ``best'' means shortest.} possible itenerary} between a set of cities, every city must be visited exactly once, and the salesman must start and finish at the same city. How can he find this itenerary?''

It is not dificult to see the practical use of solving the \TSP. 
In fact, many important problems like vehicle routing, scheduling, array clustering~\cite{tsp-tour}, and circuit design~\cite{tsp-computational-solutions} can be \emph{expressed\footnote{Formally speaking, these problems can be \emph{reduced} to \textsf{TSP.}}} as \TSP{} instances.

Furthermore, the \TSP{} is of particular theoretical interest to complexity theory researchers, as its decision variant a member of a very important family of decision problems called \textsf{NP}-complete problems.

In this document, we will introduce the \TSP, investigate some of its properties and applications, and propose a few algorithms for solving it.
