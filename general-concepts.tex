\chapter{General Concepts}




\section{Computational Complexity}

    Throughout this document, we will often find ourselevs in need of a method to objectively measure the difficulty of a problem or the efficiency of an algorithm. Fortunately, there exists an entire branch of theoretical computer science that addresses these very questions: \emph{the theory of computational complexity}.

    The theory of computational complexity formalizes the intuitive concept of the \emph{difficulty} of a problem. Quite reasonably, this discipline relies on the premise that a problem is as difficult as it is to perform its most efficient solution, or, to use technical terms, to \emph{execute the most efficient algorithm} that solves the problem.

    \subsection{Decision Problems and Complexity Classes}

For historical (and technical) reasons, most of the work in this branch has been done around a special type of problem called \emph{decision problems}. A decision problem is a problem that has a binary answer, that is, given an instance (or an input) of the problem, we compute an answer (or output) that is an element of some preknown set with cardinality 2. The sets \(\{0, 1\}\), \(\{\texttt{false}, \texttt{true}\}\), and \(\{\texttt{no}, \texttt{yes}\}\), are common examples of such a set, the former of which will be used in the rest of this discussion.

A decision problem can therefore be defined as the problem of evaluating some \emph{computable}\footnote{The model of computation is not important when defining decision problems, but it becomes so when discussing their complexity. The turing machine is the model we will use throughout this work.} function \(f: S \rightarrow \{0, 1\}\) where \(S\) is some set of inputs.

These functions can be mapped to decidable subsets of \(S\) by associating every such a set \(P\) with its characteristic function \(\mathbbm{1}_P\). This correspondence is what motivates Definition~\ref{def:decision-problem} of decidable problems.

\begin{definition}[Decision Problem]\ \\
    \label{def:decision-problem}
    A decision problem is a pair \(X = \langle S, P \rangle\) where \(S\) is a countable set called the \emph{instance space} and \(P \subset S\) is called the set of \emph{positive instances}. We say the problem \(X\) is decidable iff \(P\) is decidable (i.e. if \(\mathbbm{1}_P\) is computable).
\end{definition}

To better understand Definition~\ref{def:decision-problem}, we consider Examples~\ref{ex:decision-problems},
~and~\ref{ex:sat}, the latter of which is of particular historical significance in the context of complexity theory.
\begin{example}[A Number of Decision Problems]\ \\
    \label{ex:decision-problems}
    \begin{itemize}
        \item \textsf{PRIME} is the problem \(\langle \naturals, P\rangle\) where \(P\) is the set of all prime numbers. 
        \item \textsf{HAM}, or the \emph{hamiltonicity problem} is the problem \(\langle S, P \rangle\) where \(S\) is the set of all undirected graphs and \(P\) is the set of all hamiltonian graphs.
        \item \textsf{CLIQUE}. An instance of this problem is a pair \(\langle G, k\rangle\) where \(G\) is a graph and \(k\in\naturals\). Such an instance is positive iff \(G\) has a clique of size \(k\). 
        \item \textsf{PAIR} or the parity problem is the problem \(\langle \{0, 1\}^\ast, P \rangle\) where 
        \[P = \left\{w\in \{0, 1\}^\ast\big| |w|_1 \equiv 0 \mathrm{\ mod\ } 2 \right\}\]
    \end{itemize}
\end{example}

\begin{example}[\textsf{SAT}]\ \\
    \label{ex:sat}
    The \emph{satisfiability problem of boolean logic}, or \textsf{SAT}, is the problem \(\langle S, P\rangle\) where \(S\) is the set of all formulae of boolean logic and \(P\) is the set of all \emph{satisfiable} formulae. A formula \(\varphi \in S\) is satisfiable iff it has a satisfying assignment or a \emph{model}, that is, if its negation is not a tautology. For instance, the formula \(\varphi\) below is satisfiable (by the assignment \(x_1 = 0, x_2 = 0, x_3 = 1\) for example) while \(\psi\) is not.
    \begin{align*}
        \varphi& = \left((x_1 \vee \neg x_2) \wedge x_3\right) \vee (x_2 \wedge \neg x_3)\\
        \psi &= \neg (x_1 \vee \neg x_2) \wedge \neg (x_2 \vee \neg x_1)
    \end{align*}
\end{example}

The complexity of a decision problem is given by two pieces of information, the first being its time complexity (intuitively, this is the runtime of the \emph{fastest} turing machine deciding the problem), and the second its space complexity (the \emph{minimal} number of distinct visited cells on the tape of turing machine deciding the problem). The rigorous definitions of these quantities are given in Appendix~\ref{app:computability}. We will use the notions of \emph{algorithms}, \emph{runtime}, and \emph{memory usage} as intuitive analogues of Turing machines, runtime and space complexity respectively.

As is common when discussing complexity, we will sort problems in a hierarchy of \emph{complexity classes}. These complexity classes are based on the asymptotic behavior of the time and space complexities instead of the exact runtime or memory usage of a particular algorithm solving a particular problem. A few important complexity classes are given by Definition~\ref{def:complexity-classes}~\cite{complexity-modern}.

\begin{definition}[Most Important Complexity Classes]\ \\
    \label{def:complexity-classes}
    Let \(f: \naturals \rightarrow \reals_+\) be a function. We define the following classes of problems:
    \begin{itemize}
        \item \(\textsf{TIME}(f(n))\) is the set of problems \(X\) for which there exists a deterministic Turing machine \(\mathcal{M}\) that decides \(X\) such that \(t_{\mathcal{M}}(n) = O(f(n))\).
        \item 
        \(\textsf{NTIME}(f(n))\) is the set of problems \(X\) for which there exists a nondeterministic Turing machine \(\mathcal{M}\) that decides \(X\) such that \(t_{\mathcal{M}}(n) = O(f(n))\).    
    \end{itemize}
    From these two, we can define the following classes:
    \begin{align*}
        \textsf{P} = \bigcup\limits_{k\in\naturals} \textsf{TIME}\left(n^k\right)& &
        \textsf{NP} = \bigcup\limits_{k\in\naturals} \textsf{NTIME}\left(n^k\right) \\
        \textsf{EXPTIME} = \bigcup\limits_{k\in\naturals} \textsf{TIME}\left(2^{n^k}\right) & &
        \textsf{NEXPTIME} = \bigcup\limits_{k\in\naturals} \textsf{NTIME}\left(2^{n^k}\right) \\
    \end{align*}
\end{definition}

The list given by Definition~\ref{def:complexity-classes} is of course far from complete. In fact, a very easy way to extend it is to add for every class \(C\) its \emph{dual} class co-\({C} \colonequals \{\overline{X}| X\in C\}\) of complements of problems in \(C\).
It is however largely sufficient for the purposes of this investigation. In fact, we will mostly be dealing with the classes \textsf{P} and \textsf{NP} exclusively.

It is quite straightforward to verify the following inclusions between the classes defined thus far:
\[
    \begin{array}{ccccccc}        
        \textsf{P} &\subset& \textsf{NP} &\subset& \textsf{EXPTIME} &\subset& \textsf{NEXPTIME} \\
        \textsf{P} &\subset& \textsf{co-NP} &\subset& \textsf{EXPTIME} &\subset& \textsf{co-NEXPTIME}
    \end{array}
\]

However, it is exceedingly difficult to prove strict inclusion for most pairs. As a matter of fact, the problem of determining whether \(\textsf{P} = \textsf{NP}\) is an open one, as well as that of finding the intersection between \(C\) and co-\(C\) for \(C\in\{\textsf{NP}, \textsf{NEXPTIME}\}\). The Venn diagram of Figure~\ref{fig:classes-venn} shows the known inclusions (opting for strictness for unknown pairs).


\begin{figure}[htb]
    \begin{center}
       \includegraphics[width=10cm]{classes-venn.png}
    \end{center}
    \caption{A Venn diagram of the classes defined so far.}
    \label{fig:classes-venn}
\end{figure}


    \subsection{Reducibility and Completeness}

As we have seen in the previous section, it is currently unknown whether \(\textsf{P} = \textsf{NP}\). As far as we know, it could be as easy to decide a problem with a deterministic machine as it is to decide it with a nondeterministic one. We don't know if problems in \textsf{NP} are \emph{strictly harder} than those in \textsf{P} (even though we suspect them to be). This is hardly surprising given the vastness of \textsf{NP}. In order to prove \(\textsf{P} = \textsf{NP}\), one must find a polynomial time algorithm for \emph{every} problem in \textsf{NP}. And in order to prove that \(\textsf{P} \neq \textsf{NP}\), one must prove that for at least one problem in \textsf{NP}, \emph{all algorithms} are not polynomial time (this is admittedly an easier task than proving equality).

The point above is valid for all pairs of classes for which we have one inclusion but are uncertain of the other (\textsf{NP} and \textsf{EXPTIME} for example). The notion of completeness simplifies the task of proving inequality of two classes by reducing it to the task of proving that \emph{one particular} problem (intuitively thought of as the hardest problem) in the supposedly bigger class is in deed a member of the smaller one. These \emph{most difficult problems} are what the concept of completeness aims to formalize. In order to define them, we must first address the question of how to compare the difficulty of two problems, which is exactly what the relation of \emph{reducibility} introduced in Definition~\ref{def:polynomial-reduction} allows us to do.

\newcommand{\ple}{\preceq_\textsf{P}}
\begin{definition}[Polynomial Reduction]\ \\
    \label{def:polynomial-reduction}
    Let \(X = \langle S_1, P_1\rangle\) and \(Y =\langle S_2, P_2\rangle\) be two decision problems. A \emph{polynomial reduction} of \(X\) to \(Y\) is a function \(f:S_1 \rightarrow S_2\) computable by a deterministic Turing machine in polynomial time such that:
    \[\forall i \in S_1\ i \in P_1 \iff f(i) \in P_2\]
    If such a reduction exists, \(X\) is said to be \emph{polynomially reducible} to \(Y\), which we denote by \(X \ple Y\).
\end{definition}

One can show that \(\ple\) is both transitive, and reflexive (a preorder on decision problems) without too much difficulty. Furthermore, the relation \(\ple\) can be shown to satisfy Proposition~\ref{prop:ple-p-p} which will become very important once \textsf{NP}-completeness is defined.
\begin{proposition}\ \\
    \label{prop:ple-p-p}
    Let \(X \) and \(Y \) be two decision problems. If \(X \ple Y\) and \(Y \in \textsf{P}\), then \(X \in \textsf{P}\).
\end{proposition}
\begin{proof}\ \\
    The composition of the Turing machine that computes the polynomial reduction of \(X\) to \(Y\) and the one that decides \(Y\) in polynomial time is a polynomial time Turing machine that decides \(X\).
\end{proof}

We now define \textsf{NP}-completeness.
\begin{definition}[\textsf{NP}-hardness, \textsf{NP}-completeness]\ \\
    A problem \(X\) is \textsf{NP}-hard if and only if :
    \[\forall Y \in \textsf{NP}\ Y \ple X\]
    A problem \(X\) is \textsf{NP}-complete if and only if \(X\) is \textsf{NP}-hard and \(X\in\textsf{NP}\).
\end{definition}

Under our intuitive interpretation of \(\ple\), an \textsf{NP}-hard problem is a problem that is at least as difficult as any problem in \textsf{NP}. An \textsf{NP}-complete one is then the hardest problem in \textsf{NP}. It stands to reason then that if an \textsf{NP}-hard problem \(X\) is in \textsf{P}, we would have \(\textsf{NP} \subset \textsf{P}\) and hence \(\textsf{P}= \textsf{NP}\). This is in deed the case as affirmed by Corollary~\ref{cor:p=np}.

\begin{corollary}
    \label{cor:p=np}
    If \(\textsf{P} \cap \textsf{NP}\mathrm{-hard} \neq \emptyset\), then \(\textsf{P}=\textsf{NP}\)
\end{corollary}

\begin{proof}\ \\
    Let \(X\) be an \textsf{NP}-hard problem that is also in \textsf{P}, and let \(Y\in\textsf{NP}\). By definition of \textsf{NP}-hardness, \(Y\ple X\), and by Proposition~\ref{prop:ple-p-p} and the fact that \(X\in\textsf{P}\), \(Y\in\textsf{P}\).
    Therefore, \(\textsf{NP} \subset \textsf{P}\), and hence \(\textsf{P}=\textsf{NP}\).
\end{proof}

The same intuitive analysis that led us to Corollary~\ref{cor:p=np} suggests that if a problem is harder\footnote{In the \(\ple\) sense.} than an \textsf{NP}-hard problem, then it must be \textsf{NP}-hard as well. This is once again correct as affirmed by Proposition~\ref{prop:ple-nph}.

\begin{proposition}\ \\
    \label{prop:ple-nph}
    Let \(X\) and \(Y\) be two decision problems. If \(X\) is \textsf{NP}-hard and \(X\ple Y\), then \(Y\) is \textsf{NP}-hard.
\end{proposition}

\begin{proof}\ \\
    Let \(X, Y, Z\) be decision problems such that \(X\) is \textsf{NP}-hard and \(X\ple Y\), and \(Z \in\textsf{NP}\). By  \textsf{NP}-hardness of \(X\), \(Z \ple X\), and by transitivity of \(\ple\) we have \(Z \ple Y\). \(Y\) is then \textsf{NP}-hard.
\end{proof}

Although the idea of \textsf{NP}-completeness is promising, it is not immediately obvious how it makes approaching \textsf{P} vs \textsf{NP} easier. It would seem that proving a problem is \textsf{NP}-hard is as difficult as solving \textsf{P} vs \textsf{NP}. This is once more due to the unfathomable vastness of \textsf{NP}. Fortunately, this initial impression proved wrong. We know of many \textsf{NP}-hard and \textsf{NP}-complete problems. The first problem to be proven \textsf{NP}-complete is the problem \textsf{SAT} we invoked in Example~\ref{ex:sat}. This has been done by Stephen Cook and Leonid Levin who proved Theorem~\ref{thm:cook-levin} in 1971. A proof of this theorem is given in~\cite{langages-formels}.

\begin{theorem}[Cook Levin]\ \\
    \label{thm:cook-levin}
    \textsf{SAT} is \textsf{NP}-complete.
\end{theorem}

Now that we have one problem that we know is \textsf{NP}-complete, it becomes significantly easier to prove that a given problem is \textsf{NP}-hard. Using Proposition~\ref{prop:ple-nph}, we can show any problem is \textsf{NP}-hard by reducing a known \textsf{NP}-hard problem to it. This is what we will do in the following example by proving 3-\textsf{SAT} is \textsf{NP}-complete.

\begin{example}\ \\
    \label{ex:3sat-npc}
    An instance of 3-\textsf{SAT} is a conjunction of clauses with at most 3 literals each and is positive iff it is satisfiable. To show that 3-\textsf{SAT} is \textsf{NP}-complete, we will reduce \textsf{SAT} to it. It is technically necessary to show that \(3\textsf{-SAT}\in\textsf{NP}\) too, but this is easy given that it is a subproblem of \textsf{SAT} which is already known to be in \textsf{NP}. We will therefore focus on proving it \textsf{NP}-hard.

    To reduce \textsf{SAT} to 3-\textsf{SAT}, we must find for every formula \(\varphi\) of boolean logic a formula \(\varphi^\prime\) in 3CNF, with size polynomial in that of \(\varphi\), such that \(\varphi^\prime\) is satisfiable iff \(\varphi\) is satisfiable. We start by considering the \emph{syntax tree} of \(\varphi\), for example, the syntax tree of the formula 
    \[\varphi = \left((x_1 \vee \neg x_2) \wedge x_3\right) \vee (x_2 \wedge \neg x_3)\] 
    is shown in Figure~\ref{fig:3sat-npc-syntax-tree}. We label each internal node of this tree with a new variable (the leaves are associated with the variables of \(\varphi\)), we get the tree of Figure~\ref{fig:3sat-npc-var-tree}.

    Each new variable \(x_i\) is then associated with an equivalence \(e_i\) according to the following rules:
    \begin{itemize}
        \item If the node labeled by the variable \(x_i\) is \(\vee\), we associate it with the formula \(x_i \leftrightarrow x_j \vee x_k\) where \(x_j\) and \(x_k\) are the children of \(x_i\).
        \item If it is labeled with \(\wedge\), we associate it with the formula \(x_i \leftrightarrow x_j \wedge x_k\).
        \item If it is labeled with \(\neg\), we associate with the formula \(x_i \leftrightarrow \neg x_j\).
    \end{itemize}

    Finally, by taking the conjunction\footnote{\(n\) being the number of variables in \(\varphi\)} 
    \(\bigwedge\limits_{i>n}e_i\) of the equivalences thus obtained, and replacing equivalences with their clausal forms given by the following tautologies:
    \[
        \begin{array}{rcl}
            x \leftrightarrow \neg y &\equiv& (x\vee\neg y)\wedge (\neg x \vee y) \\
            x \leftrightarrow y \wedge z &\equiv& (x\vee\neg y\vee\neg z) \wedge 
            (\neg x \wedge y) \wedge (\neg x \wedge z) \\
            x \leftrightarrow y \wedge z &\equiv& (\neg x\vee y\vee z) \wedge 
            (x \wedge\neg y) \wedge (x \wedge\neg z) \\
        \end{array}
    \]

    We get a formula \(\varphi^\prime\) in 3CNF, which is logically equivalent to\(\varphi\) (and is a fortiori satisfiable iff \(\varphi\) is satisfiable). Furthermore, since the size of the syntax tree is proportional to the size of \(\varphi\), so is the size of \(\varphi^\prime\) (i.e. this reduction is in deed polynomial). It then follows that
    \(\textsf{SAT} \ple 3\textsf{-SAT}\), and by Proposition~\ref{prop:ple-nph}, that 3-\textsf{SAT} is \textsf{NP}-complete.
    \begin{figure}
        \begin{center}
            \begin{tikzpicture}[level 1/.style={sibling distance=30mm},level 2/.style={sibling distance=15mm}]
                \node {\(\vee\)}
                child{
                    node {\(\wedge\)}
                    child{
                        node {\(\vee\)}
                        child {node {\(x_1\)}}
                        child {
                            node {\(\neg\)}
                            child {node {\(x_2\)}}
                        }
                    }
                    child {node {\(x_3\)}}
                }
                child {
                    node {\(\wedge\)}
                    child {node {\(x_2\)}}
                    child {
                        node {\(\neg\)}
                        child {node {\(x_3\)}}
                    }
                };
            \end{tikzpicture}
        \end{center}
        \caption{Syntax tree of the formula \(\varphi\)}
        \label{fig:3sat-npc-syntax-tree}
    \end{figure}

    \begin{figure}
        \begin{center}
            \begin{tikzpicture}[level 1/.style={sibling distance=30mm},level 2/.style={sibling distance=15mm}]
                \node {\(x_4\)}
                child{
                    node {\(x_5\)}
                    child{
                        node {\(x_6\)}
                        child {node {\(x_1\)}}
                        child {
                            node {\(x_7\)}
                            child {node {\(x_2\)}}
                        }
                    }
                    child {node {\(x_3\)}}
                }
                child {
                    node {\(x_8\)}
                    child {node {\(x_2\)}}
                    child {
                        node {\(x_9\)}
                        child {node {\(x_3\)}}
                    }
                };
            \end{tikzpicture}
        \end{center}
        \caption{Syntax tree of Figure~\ref{fig:3sat-npc-syntax-tree} labeled with new variables.}
        \label{fig:3sat-npc-var-tree}
    \end{figure}
\end{example}

Using similar techniques, countless problems have been shown to be \textsf{NP}-complete. Examples include \textsf{CLIQUE} (the problem of deciding whether a graph has a \(k\)-clique for \(k\in\naturals\)),  \textsf{VERTEX-COVER} (the problem of deciding whether a graph has a covering set of vertices with cardinality \(k\) for \(k\in\naturals\)), and crucially for our study, \textsf{HAM} (mentioned in Example~\ref{ex:decision-problems}).




    % A minor technical obstacle to the use of computational complexity theory on the \TSP{}is the fact that this theory only considers so-called decision problems. The \TSP{}beeing an optimization problem, is therefore formally speaking out of the scope of this framework.

    % However, this can be metigated by associating a decision problem with the \TSP{}in such a way as to preserve its difficulty. The rest of this section will provide such an association after developing the necessary tools.

    % \subsection{Combinatorial Optimization Problems}

    %     Combinatorial optimization seeks to find an \emph{optimum} with respect to some \emph{objective} in a discrete \emph{space of options}. Therefore, it suffices to define these parameters in order to define a combinatorial optimization problem. Formally speakeing, we give the following definition:

    %     \begin{definition}[Optimization problem]\ \\
    %         A combinatorial optimization problem (or simply, an optimization problem) \(A\) is given by a quadruple \(A \colonequals (S, R, \mu, f)\) where:
    %         \begin{itemize}
    %             \item \(S\) is a finite or countably infinite set called the \emph{instance space} of \(A\).
    %             \item \(\forall x\in S\ R(x)\) is the set of \emph{feasible solutions} for the instance \(x\).
    %             \item \(\forall x\in S, \forall y \in R(x)\ \mu(x, y)\in\reals_+\) is the measure of \(y\).
    %         \end{itemize}
    %     \end{definition}

    %     Using the above definition, \TSP{}can be defined as \(\TSP{}= (S, R, \mu, f)\) with:
    %     \begin{itemize}
    %         \item \(S\) the set of all complete weighted graphs.
    %         \item For \(G\in S\), \(R(G)\) is the set of all hamiltonian cycles in \(G\).
    %         \item For a hamiltonian cycle \(C\) of \(G\), \(\mu(G, C)\) is the length of \(C\).
    %         \item Finally, the problem is to minimize \(\mu(G, C)\).
    %     \end{itemize}


    % \subsection{Decision Problems}

    %     In contrast to optimization problems, which can have a very large if finite space of solutions, an instance of a decision problem has a binary answer. Hence, decision problems are significantly easier to define and analyze from a theoretical point of view. Formally speaking, we define a decision problem as:

    %     \begin{definition}[Decision problem]\ \\
    %         A decision problem \(A\) is given by a pair \(A \colonequals (S, B)\) where:
    %         \begin{itemize}
    %             \item \(S\) is a set called the \emph{instance space} of \(A\).
    %             \item \(B \subset S\) is the set of \emph{positive instances} of \(A\) (that is instances for which the answer is yes).
    %         \end{itemize}
    %     \end{definition}