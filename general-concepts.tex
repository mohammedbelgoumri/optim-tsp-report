\chapter{General Concepts}




\section{Computational Complexity}

    Throughout this document, we will often find ourselevs in need of a method to objectively measure the difficulty of a problem or the efficiency of an algorithm. Fortunately, there exists an entire branch of theoretical computer science that addresses these very questions: \emph{the theory of computational complexity}.

    The theory of computational complexity formalizes the intuitive concept of the \emph{difficulty} of a problem. Quite reasonably, this discipline relies on the premise that a problem is as difficult as it is to perform its most efficient solution, or, to use technical terms, to \emph{execute the most efficient algorithm} that solves the problem.

    \subsection{Decision Problems and Complexity Classes}

For historical (and technical) reasons, most of the work in this branch has been done around a special type of problem called \emph{decision problems}. A decision problem is a problem that has a binary answer, that is, given an instance (or an input) of the problem, we compute an answer (or output) that is an element of some preknown set with cardinality 2. The sets \(\{0, 1\}\), \(\{\texttt{false}, \texttt{true}\}\), and \(\{\texttt{no}, \texttt{yes}\}\), are common examples of such a set, the former of which will be used in the rest of this discussion.

A decision problem can therefore be defined as the problem of evaluating some \emph{computable}\footnote{The model of computation is not important when defining decision problems, but it becomes so when discussing their complexity. The turing machine is the model we will use throughout this work.} function \(f: S \rightarrow \{0, 1\}\) where \(S\) is some set of inputs.

These functions can be mapped to decidable subsets of \(S\) by associating every such a set \(P\) with its characteristic function \(\mathbbm{1}_P\). This correspondence is what motivates Definition~\ref{def:decision-problem} of decidable problems.

\begin{definition}[Decision Problem]\ \\
    \label{def:decision-problem}
    A decision problem is a pair \(X = \langle S, P \rangle\) where \(S\) is a countable set called the \emph{instance space} and \(P \subset S\) is called the set of \emph{positive instances}. We say the problem \(X\) is decidable iff \(P\) is decidable (i.e. if \(\mathbbm{1}_P\) is computable).
\end{definition}

To better understand Definition~\ref{def:decision-problem}, we consider Examples~\ref{ex:decision-problems},
~and~\ref{ex:sat}, the latter of which is of particular historical significance in the context of complexity theory.
\begin{example}[A Number of Decision Problems]\ \\
    \label{ex:decision-problems}
    \begin{itemize}
        \item \textsf{PRIME} is the problem \(\langle \naturals, P\rangle\) where \(P\) is the set of all prime numbers. 
        \item \textsf{HAM}, or the \emph{hamiltonicity problem} is the problem \(\langle S, P \rangle\) where \(S\) is the set of all undirected graphs and \(P\) is the set of all hamiltonian graphs.
        \item \textsf{CLIQUE}. An instance of this problem is a pair \(\langle G, k\rangle\) where \(G\) is a graph and \(k\in\naturals\). Such an instance is positive iff \(G\) has a clique of size \(k\). 
        \item \textsf{PAIR} or the parity problem is the problem \(\langle \{0, 1\}^\ast, P \rangle\) where 
        \[P = \left\{w\in \{0, 1\}^\ast\big| |w|_1 \equiv 0 \mathrm{\ mod\ } 2 \right\}\]
    \end{itemize}
\end{example}

\begin{example}[\textsf{SAT}]\ \\
    \label{ex:sat}
    The \emph{satisfiability problem of boolean logic}, or \textsf{SAT}, is the problem \(\langle S, P\rangle\) where \(S\) is the set of all formulae of boolean logic and \(P\) is the set of all \emph{satisfiable} formulae. A formula \(\varphi \in S\) is satisfiable iff it has a satisfying assignment or a \emph{model}, that is, if its negation is not a tautology. For instance, the formula \(\varphi\) below is satisfiable (by the assignment \(x_1 = 0, x_2 = 0, x_3 = 1\) for example) while \(\psi\) is not.
    \begin{align*}
        \varphi& = \left((x_1 \vee \neg x_2) \wedge x_3\right) \vee (x_2 \wedge \neg x_3)\\
        \psi &= \neg (x_1 \vee \neg x_2) \wedge \neg (x_2 \vee \neg x_1)
    \end{align*}
\end{example}

The complexity of a decision problem is given by two pieces of information, the first being its time complexity (intuitively, this is the runtime of the \emph{fastest} turing machine deciding the problem), and the second its space complexity (the \emph{minimal} number of distinct visited cells on the tape of turing machine deciding the problem). The rigorous definitions of these quantities are given in Appendix~\ref{app:computability}. We will use the notions of \emph{algorithms}, \emph{runtime}, and \emph{memory usage} as intuitive analogues of Turing machines, runtime and space complexity respectively.

As is common when discussing complexity, we will sort problems in a hierarchy of \emph{complexity classes}. These complexity classes are based on the asymptotic behavior of the time and space complexities instead of the exact runtime or memory usage of a particular algorithm solving a particular problem. A few important complexity classes are given by Definition~\ref{def:complexity-classes}~\cite{complexity-modern}.

\begin{definition}[Most Important Complexity Classes]\ \\
    \label{def:complexity-classes}
    Let \(f: \naturals \rightarrow \reals_+\) be a function. We define the following classes of problems:
    \begin{itemize}
        \item \(\textsf{TIME}(f(n))\) is the set of problems \(X\) for which there exists a deterministic Turing machine \(\mathcal{M}\) that decides \(X\) such that \(t_{\mathcal{M}}(n) = O(f(n))\).
        \item 
        \(\textsf{NTIME}(f(n))\) is the set of problems \(X\) for which there exists a nondeterministic Turing machine \(\mathcal{M}\) that decides \(X\) such that \(t_{\mathcal{M}}(n) = O(f(n))\).    
    \end{itemize}
    From these two, we can define the following classes:
    \begin{align*}
        \textsf{P} = \bigcup\limits_{k\in\naturals} \textsf{TIME}\left(n^k\right)& &
        \textsf{NP} = \bigcup\limits_{k\in\naturals} \textsf{NTIME}\left(n^k\right) \\
        \textsf{EXPTIME} = \bigcup\limits_{k\in\naturals} \textsf{TIME}\left(2^{n^k}\right) & &
        \textsf{NEXPTIME} = \bigcup\limits_{k\in\naturals} \textsf{NTIME}\left(2^{n^k}\right) \\
    \end{align*}
\end{definition}

The list given by Definition~\ref{def:complexity-classes} is of course far from complete. In fact, a very easy way to extend it is to add for every class \(C\) its \emph{dual} class co-\({C} \colonequals \{\overline{X}| X\in C\}\) of complements of problems in \(C\).
It is however largely sufficient for the purposes of this investigation. In fact, we will mostly be dealing with the classes \textsf{P} and \textsf{NP} exclusively.

It is quite straightforward to verify the following inclusions between the classes defined thus far:
\[
    \begin{array}{ccccccc}        
        \textsf{P} &\subset& \textsf{NP} &\subset& \textsf{EXPTIME} &\subset& \textsf{NEXPTIME} \\
        \textsf{P} &\subset& \textsf{co-NP} &\subset& \textsf{EXPTIME} &\subset& \textsf{co-NEXPTIME}
    \end{array}
\]

However, it is exceedingly difficult to prove strict inclusion for most pairs. As a matter of fact, the problem of determining whether \(\textsf{P} = \textsf{NP}\) is an open one, as well as that of finding the intersection between \(C\) and co-\(C\) for \(C\in\{\textsf{NP}, \textsf{NEXPTIME}\}\). The Venn diagram of Figure~\ref{fig:classes-venn} shows the known inclusions (opting for strictness for unknown pairs).


\begin{figure}
    \begin{center}
       \includegraphics[width=10cm]{classes-venn.png}
    \end{center}
    \caption{A Venn diagram of the classes defined so far.}
    \label{fig:classes-venn}
\end{figure}


    \subsection{Reducibility and Completeness}

As we have seen in the previous section, it is currently unknown whether \(\textsf{P} = \textsf{NP}\). As far as we know, it could be as easy to decide a problem with a deterministic machine as to decide it with a nondeterministic machine. We don't know if problems in \textsf{NP} are \emph{strictly harder} thans those in \textsf{P} (even though we suspect that they are). This is hardly surprising given the vastness of \textsf{NP}. In order to prove \(\textsf{P} = \textsf{NP}\), one must find a polynomial time algorithm for \emph{every} problem in \textsf{NP}. And in order to prove that \(\textsf{P} \neq \textsf{NP}\), one must prove that for at least one problem in \textsf{NP}, \emph{all algorithms} are not polynomial time (this is admittedly an easier task than proving equality).

The point above is valid for all pairs of classes for which we have one inclusion but are uncertain of the other (\textsf{NP} and \textsf{EXPTIME} for example). The notio of complexity tries to simplify the task of proving inequality of two classes by reducing it to the task of proving that \emph{one particular} problem (intuitively thought of as the hardest problem in that class) of the supposedly bigger class is actually in the smaller one. These \emph{most difficult problems} are what the concept of completeness aims to formalize. In order to define them, we must first address the question of how to compare the difficulty of two problems, which is exactly what the relation of \emph{reducibility} introduced in Definition~\ref{def:polynomial-reduction} attempts to do.

\newcommand{\ple}{\preceq_\textsf{P}}
\begin{definition}[Polynomial Reduction]\ \\
    \label{def:polynomial-reduction}
    Let \(X = \langle S_1, P_1\rangle\) and \(Y =\langle S_2, P_2\rangle\) be two decision problems. A \emph{polynomial reduction} of \(X\) to \(Y\) is a function \(f:S_1 \rightarrow S_2\) computable by a deterministic Turing machine in polynomial time such that:
    \[\forall i \in S_1\ i \in P_1 \iff f(i) \in P_2\]
    If such a reduction exists, \(X\) is said to be \emph{polynomially reducible} to \(Y\), which we denote by \(X \ple Y\).
\end{definition}



    % A minor technical obstacle to the use of computational complexity theory on the \TSP{}is the fact that this theory only considers so-called decision problems. The \TSP{}beeing an optimization problem, is therefore formally speaking out of the scope of this framework.

    % However, this can be metigated by associating a decision problem with the \TSP{}in such a way as to preserve its difficulty. The rest of this section will provide such an association after developing the necessary tools.

    % \subsection{Combinatorial Optimization Problems}

    %     Combinatorial optimization seeks to find an \emph{optimum} with respect to some \emph{objective} in a discrete \emph{space of options}. Therefore, it suffices to define these parameters in order to define a combinatorial optimization problem. Formally speakeing, we give the following definition:

    %     \begin{definition}[Optimization problem]\ \\
    %         A combinatorial optimization problem (or simply, an optimization problem) \(A\) is given by a quadruple \(A \colonequals (S, R, \mu, f)\) where:
    %         \begin{itemize}
    %             \item \(S\) is a finite or countably infinite set called the \emph{instance space} of \(A\).
    %             \item \(\forall x\in S\ R(x)\) is the set of \emph{feasible solutions} for the instance \(x\).
    %             \item \(\forall x\in S, \forall y \in R(x)\ \mu(x, y)\in\reals_+\) is the measure of \(y\).
    %         \end{itemize}
    %     \end{definition}

    %     Using the above definition, \TSP{}can be defined as \(\TSP{}= (S, R, \mu, f)\) with:
    %     \begin{itemize}
    %         \item \(S\) the set of all complete weighted graphs.
    %         \item For \(G\in S\), \(R(G)\) is the set of all hamiltonian cycles in \(G\).
    %         \item For a hamiltonian cycle \(C\) of \(G\), \(\mu(G, C)\) is the length of \(C\).
    %         \item Finally, the problem is to minimize \(\mu(G, C)\).
    %     \end{itemize}


    % \subsection{Decision Problems}

    %     In contrast to optimization problems, which can have a very large if finite space of solutions, an instance of a decision problem has a binary answer. Hence, decision problems are significantly easier to define and analyze from a theoretical point of view. Formally speaking, we define a decision problem as:

    %     \begin{definition}[Decision problem]\ \\
    %         A decision problem \(A\) is given by a pair \(A \colonequals (S, B)\) where:
    %         \begin{itemize}
    %             \item \(S\) is a set called the \emph{instance space} of \(A\).
    %             \item \(B \subset S\) is the set of \emph{positive instances} of \(A\) (that is instances for which the answer is yes).
    %         \end{itemize}
    %     \end{definition}