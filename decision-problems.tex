\subsection{Decision Problems and Complexity Classes}

For historical (and technical) reasons, most of the work in this branch has been done around a special type of problem called \emph{decision problems}. A decision problem is a problem that has a binary answer, that is, given an instance (or an input) of the problem, we compute an answer (or output) that is an element of some preknown set with cardinality 2. The sets \(\{0, 1\}\), \(\{\texttt{false}, \texttt{true}\}\), and \(\{\texttt{no}, \texttt{yes}\}\), are common examples of such a set, the former of which will be used in the rest of this discussion.

A decision problem can therefore be defined as the problem of evaluating some \emph{computable}\footnote{The model of computation is not important when defining decision problems, but it becomes so when discussing their complexity. The turing machine is the model we will use throughout this work.} function \(f: S \rightarrow \{0, 1\}\) where \(S\) is some set of inputs.

These functions can be mapped to decidable subsets of \(S\) by associating every such a set \(P\) with its characteristic function \(\mathbbm{1}_P\). This correspondence is what motivates Definition~\ref{def:decision-problem} of decidable problems.

\begin{definition}[Decision Problem]\ \\
    \label{def:decision-problem}
    A decision problem is a pair \(X = \langle S, P \rangle\) where \(S\) is a countable set called the \emph{instance space} and \(P \subset S\) is called the set of \emph{positive instances}. We say the problem \(X\) is decidable iff \(P\) is decidable (i.e. if \(\mathbbm{1}_P\) is computable).
\end{definition}

To better understand Definition~\ref{def:decision-problem}, we consider Examples~\ref{ex:decision-problems},
~and~\ref{ex:sat}, the latter of which is of particular historical significance in the context of complexity theory.
\begin{example}[A Number of Decision Problems]\ \\
    \label{ex:decision-problems}
    \begin{itemize}
        \item \textsf{PRIME} is the problem \(\langle \naturals, P\rangle\) where \(P\) is the set of all prime numbers. 
        \item \textsf{HAM}, or the \emph{hamiltonicity problem} is the problem \(\langle S, P \rangle\) where \(S\) is the set of all undirected graphs and \(P\) is the set of all hamiltonian graphs.
        \item \textsf{CLIQUE}. An instance of this problem is a pair \(\langle G, k\rangle\) where \(G\) is a graph and \(k\in\naturals\). Such an instance is positive iff \(G\) has a clique of size \(k\). 
        \item \textsf{PAIR} or the parity problem is the problem \(\langle \{0, 1\}^\ast, P \rangle\) where 
        \[P = \left\{w\in \{0, 1\}^\ast\big| |w|_1 \equiv 0 \mathrm{\ mod\ } 2 \right\}\]
    \end{itemize}
\end{example}

\begin{example}[\textsf{SAT}]\ \\
    \label{ex:sat}
    The \emph{satisfiability problem of boolean logic}, or \textsf{SAT}, is the problem \(\langle S, P\rangle\) where \(S\) is the set of all formulae of boolean logic and \(P\) is the set of all \emph{satisfiable} formulae. A formula \(\varphi \in S\) is satisfiable iff it has a satisfying assignment or a \emph{model}, that is, if its negation is not a tautology. For instance, the formula \(\varphi\) below is satisfiable (by the assignment \(x_1 = 0, x_2 = 0, x_3 = 1\) for example) while \(\psi\) is not.
    \begin{align*}
        \varphi& = \left((x_1 \vee \neg x_2) \wedge x_3\right) \vee (x_2 \wedge \neg x_3)\\
        \psi &= \neg (x_1 \vee \neg x_2) \wedge \neg (x_2 \vee \neg x_1)
    \end{align*}
\end{example}

The complexity of a decision problem is given by two pieces of information, the first being its time complexity (intuitively, this is the runtime of the \emph{fastest} turing machine deciding the problem), and the second its space complexity (the \emph{minimal} number of distinct visited cells on the tape of turing machine deciding the problem). The rigorous definitions of these quantities are given in Appendix~\ref{app:computability}. We will use the notions of \emph{algorithms}, \emph{runtime}, and \emph{memory usage} as intuitive analogues of Turing machines, runtime and space complexity respectively.

As is common when discussing complexity, we will sort problems in a hierarchy of \emph{complexity classes}. These complexity classes are based on the asymptotic behavior of the time and space complexities instead of the exact runtime or memory usage of a particular algorithm solving a particular problem. A few important complexity classes are given by Definition~\ref{def:complexity-classes}~\cite{complexity-modern}.

\begin{definition}[Most Important Complexity Classes]\ \\
    \label{def:complexity-classes}
    Let \(f: \naturals \rightarrow \reals_+\) be a function. We define the following classes of problems:
    \begin{itemize}
        \item \(\textsf{TIME}(f(n))\) is the set of problems \(X\) for which there exists a deterministic Turing machine \(\mathcal{M}\) that decides \(X\) such that \(t_{\mathcal{M}}(n) = O(f(n))\).
        \item 
        \(\textsf{NTIME}(f(n))\) is the set of problems \(X\) for which there exists a nondeterministic Turing machine \(\mathcal{M}\) that decides \(X\) such that \(t_{\mathcal{M}}(n) = O(f(n))\).    
    \end{itemize}
    From these two, we can define the following classes:
    \begin{align*}
        \textsf{P} = \bigcup\limits_{k\in\naturals} \textsf{TIME}\left(n^k\right)& &
        \textsf{NP} = \bigcup\limits_{k\in\naturals} \textsf{NTIME}\left(n^k\right) \\
        \textsf{EXPTIME} = \bigcup\limits_{k\in\naturals} \textsf{TIME}\left(2^{n^k}\right) & &
        \textsf{NEXPTIME} = \bigcup\limits_{k\in\naturals} \textsf{NTIME}\left(2^{n^k}\right) \\
    \end{align*}
\end{definition}

The list given by Definition~\ref{def:complexity-classes} is of course far from complete. In fact, a very easy way to extend it is to add for every class \(C\) its \emph{dual} class co-\({C} \colonequals \{\overline{X}| X\in C\}\) of complements of problems in \(C\).
It is however largely sufficient for the purposes of this investigation. In fact, we will mostly be dealing with the classes \textsf{P} and \textsf{NP} exclusively.

It is quite straightforward to verify the following inclusions between the classes defined thus far:
\[
    \begin{array}{ccccccc}        
        \textsf{P} &\subset& \textsf{NP} &\subset& \textsf{EXPTIME} &\subset& \textsf{NEXPTIME} \\
        \textsf{P} &\subset& \textsf{co-NP} &\subset& \textsf{EXPTIME} &\subset& \textsf{co-NEXPTIME}
    \end{array}
\]

However, it is exceedingly difficult to prove strict inclusion for most pairs. As a matter of fact, the problem of determining whether \(\textsf{P} = \textsf{NP}\) is an open one, as well as that of finding the intersection between \(C\) and co-\(C\) for \(C\in\{\textsf{NP}, \textsf{NEXPTIME}\}\). The Venn diagram of Figure~\ref{fig:classes-venn} shows the known inclusions (opting for strictness for unknown pairs).


\begin{figure}
    \begin{center}
       \includegraphics[width=10cm]{classes-venn.png}
    \end{center}
    \caption{A Venn diagram of the classes defined so far.}
    \label{fig:classes-venn}
\end{figure}

