\chapter{The Traveling Salesman Problem}

    \section{History}

    The first use of the term 'traveling salesman problem' in mathematical
    circles may have been in 1931-32, as we shall explain below. But in 1832, a
    book was printed in Germany entitled Der Handlungsreisende, wie er sein
    soil und was er zu thun hat, um Auftrage zu erhalten und eines glucklichen
    Erfolgs in seines Geschdften gewiss zu sein. Von einem alien Commis-
    Voyageur ("The Traveling Salesman, how he should be and what he should
    do to get Commissions and to be Successful in his Business. By a veteran
    Traveling Salesman'). 
    
    Although devoted for the most part to other issues,
    the book reaches the essence of the TSP in its last chapter: 'By a proper
    choice and scheduling of the tour, one can often gain so much time that we
    have to make some suggestions.... The most important aspect is to cover as
    many locations as possible without visiting a location twice ...' [Voigt, 1831;
    MiMer-Merbach, 1983].

    \section{Motivation}

    \section{Formal Statement}