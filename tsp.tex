\chapter{The Traveling Salesman Problem}

    \section{A Brief History}

    The mathematical study of the problem we have been referring to as the Traveling Salesman Problem most likely began in the 1930s~\cite{tsp-computational-solutions}. However, the origins of that terminology remain unclear.

    What is outside the realm of debate, is the fact that the problem itself has been popularized among mathematicians thanks to the efforts of mathematician Merrill Flood~\cite{tsp-computational-study}. He introduced it to his colleagues at the RAND corporation\footnote{Research and Development, An American think tank created in 1948.}, who interned made it popular in the wider world of operations research to the point that it became the archetypical example of a hard combinatorial optimization problem.

    Although, it should be noted that in 1831, a non-mathematical text by the title \emph{``Der Handlungsreisende, wie er sein soil und was er zu thun hat, um Auftrage zu erhalten und eines glucklichen Erfolgs in seines Geschdften gewiss zu sein''}, which translates to \emph{``The Traveling Salesman, how he should be and what he should do to get Commissions and to be Successful in his Business. By a veteran Traveling Salesman''}~\cite{tsp-tour,original}. In the last chapter, one can read 
    \begin{quote}
        By a proper choice and scheduling of the tour, one can often gain so much time that we have to make some suggestions \dots{} The most important aspect is to cover as many locations as possible without visiting a location twice \dots
    \end{quote}

    In 1954, the paper ``Solution of a Large-Scale Traveling-Salesman Problem''~\cite{large-scale} which caused interest in the \TSP to explode.

    \section{Motivation}

    \section{Formal Statement}